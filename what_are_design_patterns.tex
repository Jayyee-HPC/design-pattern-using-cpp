\documentclass[11pt]{article}
\usepackage[utf8]{inputenc}	% Para caracteres en español
\usepackage{amsmath,amsthm,amsfonts,amssymb,amscd}
\usepackage{multirow,booktabs}
\usepackage[table]{xcolor}
\usepackage{fullpage}
\usepackage{lastpage}
\usepackage{enumitem}
\usepackage{fancyhdr}
\usepackage{mathrsfs}
\usepackage{wrapfig}
\usepackage{setspace}
\usepackage{calc}
\usepackage{multicol}
\usepackage{cancel}
\usepackage[retainorgcmds]{IEEEtrantools}
\usepackage[margin=3cm]{geometry}
\usepackage{amsmath}
\newlength{\tabcont}
\setlength{\parindent}{0.0in}
\setlength{\parskip}{0.05in}
\usepackage{empheq}
\usepackage{framed}
\usepackage[most]{tcolorbox}
\usepackage{xcolor}
\colorlet{shadecolor}{orange!15}
\parindent 0in
\parskip 12pt
\geometry{margin=1in, headsep=0.25in}
\theoremstyle{definition}
\newtheorem{defn}{Definition}
\newtheorem{reg}{Rule}
\newtheorem{exer}{Exercise}
\newtheorem{note}{Note}
\begin{document}
%\setcounter{section}{8}
%\title{Chapter 9 Review Notes}

\thispagestyle{empty}

\begin{center}
{\LARGE \bf What Are Design Patterns}\\
05 2022
\end{center}

\section{Four Essential Elements}

\begin{enumerate}
    \item The \textbf{pattern name} is a handle we can use to describe a desgin problem, its solutions, and consequences in a word or two.
        Naming a pattern immediately increases our design vocabulary.
        It lets us design at a higher level of abstraction.
        Having a vocabulary for patterns lets us talk about them with our colleagues, in our documentation, and even to ourselves.
        It makes it easier to think about designs and to communicate them and their trade-offs to others.
        Finding good names has been of the hardest parts of developing our catalog.

    \item The \textbf{problem} describes when to apply the pattern.
        It explains the problem and its context.
        It might describe specific design problems such as how to represent algorithms as objects.
        It might describe class or object structures that are symptomatic of an inflexible design.
        Sometimes the problem will include a list of conditions that must be met before it makes sense to apply the pattern.
    
    \item The \textbf{solution} describes the elements that make up the design, their relationships, responsibilities, and collaborations.
        The solution doesn't describe a particular concret design or implementation, because a pattern is like a template that can be applied in many different situations.
        Instead, the pattern provides an abstract description of a design problem and how a genereal arrangement of elements (classes and objects in our case) solves it.

    \item The \textbf{consequences} are the results and trade-offs of applying the pattern.
        Though consequences are often unvoiced when we describe design decisions, they are critical for evaluating design alternatives and for understanding the costs and benefits of applying the pattern.
        The consequences for software often concern space and time trade-offs.
        They may address language and implementaion issues as well. 
        Since reuse is often a factor in object-oriented desgin, the consequences of a pattern include its impact on a system's flexibility, extensibility, or portability.
        Listing these consequences explicitly helps you understand and evaluate them.
\end{enumerate}

\section{Mechanics in Non-inertial Reference Frames}
\subsection{Acceleration Without Rotation}
Consider an inertial reference frame (i.e not accelerating) which will be denoted S$_0$, and a accelerating reference frame, \textit{S} that has an acceleration of \textit{A}. 
\begin{note}
\textbf{Capital Letters refer to the accelerating reference frame \textit{S} while lowercase letters refer to the inertial reference frame S$_0$}
\end{note}
Picture a moving reference frame, \textit{S}, moving relative to S$_0$. Imagine in the the moving reference frame that a ball with mass, \textit{m} is being thrown. 
In order to consider the motion of the ball, the motion must be first considered in the inertial reference frame. 
\begin{equation}
F = m\ddot{r_0}
\end{equation}
Where r$_0$ is the ball's position relative to S$_0$. 

Now, by considering the motion of the ball in the accelerating frame, the ball position relative to \textit{S} is \textit{R}. (It's velocity is $\dot{R}$. 
Thus, relating \textit{R} to $r_0$, we have: 
\begin{equation}
\dot{r_0} = \dot{R} + V
\end{equation}
Newton's second law for the inertial reference frame by differentiate and multiplying by mass is:
\begin{equation}
F_{\text{inertial}} = -mA = -m\ddot{R}
\end{equation}
\subsection{The Tides}
\begin{shaded}
\textbf{The Tidal Force} \newline
\begin{equation}
F_{tide} = -GM_mm(\frac{\hat{d}}{d^2}-\frac{\hat{d_0}}{d_0^2})
\end{equation}
Where:
\begin{equation*}
\begin{split}
G = \text{Gravitational Constant} \\
d = \text{Object's Position Relative to Moon} \\
d_0 = \text{Earth's Center Relative to the moon}\\
M_m = \text{Mass of the moon}
\end{split}
\end{equation*}
\end{shaded}
\newpage
\subsection{The Angular Velocity Vector}
The rest of the notes and the chapter will over reference frames that are rotating with respect to the inertial reference frame, so angular velocity has to be used. 
\begin{defn}
\textbf{Euler's Theorem} - The most general motion of any body relative to a fixed point \textit{O} is a rotation about some axis through \textit{O} To specify this rotation about a given point O, we only have to give the direction of the axis and the rate of rotation, or angular velocity $\omega$. Because this has a magnitude and direction, it is an obvious choice to write this rotation vector as $\omega$, the angular velocity vector. That is:
\begin{equation}
\omega = \omega\textbf{u}
\end{equation}
Where \textbf{u} is the unit vector
\end{defn}
\begin{shaded}
\textbf{Vector Velocity}\newline
The velocity at any point, \textit{P} (position, \textit{r}) is given by:
\begin{equation}
v = \omega\  x \ r
\end{equation}
\end{shaded}
\subsection*{Addition of Angular Velocities}
One can add angular velocities just like linear velocities. If body 3 is rotating at angular velocity $\omega_{32}$ relative to frame 2, and frame 2 is rotating at angular velocity $\omega_{21}$ relative to frame 1, then body 3 is rotating relative to frame 1 at angular velocity: 
\begin{equation}
\omega_{31} = \omega_{32} + \omega_{21}
\end{equation}
\subsection{Time Derivatives in Rotating Frames}
If frame S has a angular velocity, $\Omega$ relative to S$_0$ then the time derivative of a single vector \textbf{Q} as seen in the two frames are related by:
\begin{equation}
(\frac{d\textbf{Q}}{dt})_{S_0} = (\frac{d\textbf{Q}}{dt})_{S} \ + \Omega \ x \ \textbf{Q}
\end{equation}
\subsection{Netwon's Second Law in a Rotating Frame}
A particle in an inertial reference frame, S$_0$ obeys Newton's second law as we are use to:
\begin{equation}
m\frac{d^2r}{dt^2} = F
\end{equation}
Using the results from equation 8, the time derivative for a rotating frame with reference to an inertial frame can be given by:
\begin{equation}
(\frac{dr}{dt})_{S_0} = (\frac{dr}{dt})_s \ + \Omega \ x \ r
\end{equation}
By differentiation, Newton's second law becomes:
\begin{equation}
m\ddot{r} = F + 2m\dot{r} \ x \ \Omega \ + m(\Omega \ x \ r) \ x \ \Omega
\end{equation}
Where \textit{F} is the sum of all forces in the inertial reference frame. 
\subsection{The Centrifugal Force}
This is an inertial force in a rotating reference frame 
\begin{equation}
F_{\text{cf}} = m(\Omega \ x \ r) \ x \ \Omega
\end{equation}
\subsubsection*{Free-Fall Acceleration (Non-Vertical Gravity)}
\begin{equation}
F_{\text{eff}} = F_{\text{grav}} + F_{\text{cf}} = mg_0 + m\Omega^2R\sin(\theta)\hat{\rho}
\end{equation}
The acceleration due to the Centrifugal force is simply 
\begin{equation}
\begin{split}
g = g_0 + \Omega^2R\sin(\theta)\hat{\rho} \\
g_{\text{rad}} = g_0 - \Omega^2R\sin^2(\theta)  \\
g_{\text{tan}} = \Omega^2R\sin(\theta)\cos(\theta)
\end{split}
\end{equation}
The angle between g and its radial direction is:
\begin{equation}
\alpha \approx \frac{g_{\text{tan}}}{g_{\text{rad}}} 
\end{equation}
The maxium value at ($\theta$ = 45):
\begin{equation}
\alpha_{\text{max}} =  \frac{\Omega^2R}{2g_0}
\end{equation}
\subsection{Coriolis Force}
The Coriolis Force is another inertial force in a rotating reference frame that an object experiences when it is moving. 
\begin{equation}
F_{\text{cor}} = 2m\dot{r} \ x \ \Omega = 2mv \ x \ \Omega
\end{equation}
The maximum acceleration, \textit{a} that the Coriolis force could produce acting by itself with \textit{v} perpendicular to $\Omega$ is:
\begin{equation}
a_{\text{max}} = 2v\Omega 
\end{equation}
\begin{shaded}
\textbf{Direction of the Coriolis Force} \newline
The Direction of the Coriolis force us always perpendicular to the velocity of the object (hence equation 17), and is given by the right hand rule. 
\end{shaded}
\newpage
\subsection{Free Fall and the Coriolis Force}
\begin{equation}
m\ddot{r} = mg_0 + F_{\text{cf}} + F_{\text{cor}} 
\end{equation}
\subsection{The Foucault Pendulum}
See chapter 9, Page 354. There is no need to recopy what is in the book here. 

\end{document}